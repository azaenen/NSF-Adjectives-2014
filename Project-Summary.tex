\documentclass[10pt]{article}

\usepackage{fullpage}
\usepackage{palatino}


\begin{document}

\begin{center}
\vspace{-0.5em}
{\Large {\bf Project Summary}}\\*[3mm]
\vspace{-0.5em}
{\bf IIS-RI MEDIUM: 
Lexical Inference Patterns for Adjectives in Natural Language}\\
{\bf PI: James Pustejovsky,  Inst: Brandeis University}; {\bf PI: Christiane Fellbaum,  Inst: Princeton University};
{\bf PI: Cleo Condoravdi,  Inst: Stanford University}; {\bf PI: Daniel Lassiter,  Inst: Stanford University}
\end{center}


\vspace{-0.5em}
Effective communication relies heavily on the ability of language users to recover information that is not explicitly expressed in utterances. 
Much of the covert information can be identified as ``semantic inferences", and as such, be associated with identifiable structural or lexical patterns in natural language.
An understanding of how speakers identify and exploit systematic covert inferences in language has the potential to 
enrich our models of compositionally derived inferences. 
At the same time, it can enhance the capabilities of natural language understanding systems to read beyond the surface forms of the text, a major goals of current {\sc nlp} research. 
We address the ways speakers interpret texts with respect to the following questions: Did events referred to in the text in fact occur or not? 
Are they desirable? 
How do speakers interpret qualities ascribed to entities in a text? 
To answer subtle questions such as these, careful analysis of the lexical semantics of adjectives is needed. 
We propose to address the current lack of readily exploitable linguistic information for this category 
and to explore its modeling in a large-scale, empirical investigation considering the semantic interactions 
among textual elements as well as the linguistic judgments of ``naive" speakers. The specific aims of this proposal are:

\vspace{-0.4em}
\begin{itemize}

\item Develop an inferential model for adjectival semantics in natural language;

\vspace{-0.5em}
\item Connect this model to data by formulating templates of structure-to-inference mappings using data mining techniques over Web corpora. 

\vspace{-0.5em}
\item Revise and enrich the  theoretical model and inference templates by examining the same data ``in the wild", that is,  crowdsourced judgments using larger textual contexts. 
\end{itemize}


\vspace{-.5em}

We concentrate on three diverse classes of  adjectives, 
in order to both: (a) test the applicability of the methodology to different linguistic classes; and (b) to articulate just how the structure-to-inference mapping can be modeled within each lexical class. 
The adjective types studied are: 
(i) dimensional and evaluative adjectives with scalar values and associated scalar implicatures, e.g., \textit{pretty, beautiful, large, huge}; (ii) evidentiality adjectives, showing varying implicatures of veridicity over a clausal complement, e.g., \textit{rude, annoying, likely}, etc.; and (iii) intensional adjectives, introducing implicatures of modal subordination, e.g., \textit{alleged, supposed, so-called}. 
The work will result in a small Gold standard inference  corpus created by using a standard linguistic annotation effort following explicit guidelines indicating the structure-to-inference mapping for each type of adjective. 
We compare these baseline mappings to inferential judgments made by naive native speakers (Mechanical Turk workers). 
Preliminary studies suggest a variance from the baseline, caused by textual factors abstracted away in linguistic studies but important to explain naive judgments. 
We use these differential judgments (linguist vs. naive annotator) 
to classify the implicatures along two dimensions: (1) how stable an inference is regardless of linguistic context; (2) which contextual factors contribute to blocking the expected inference. 
We construct a model to gauge how well our distinctions explain the behavior of these speakers. This provides an account for the interactions of different structural and lexical factors. 

\vspace{-1.2em}
\paragraph{Intellectual Merit.} The proposed work 
develops and validates a methodology for the empirical discovery and exploitation of systematic inferences identified with three distinct classes of adjectives in natural language
and provides a novel and subtle analysis of these adjectives with sensitivity to their linguistic contexts. An important contribution will be a richer conceptualization 
of corpus and lexicon annotation with significant consequences for computational linguists. 

\vspace{-1.2em}
\paragraph{Broader Impact.}
The proposed work  makes several significant contributions to a broader community of computational linguists and AI researchers. 
One is to lay the groundwork for the 
large-scale annotation of three classes of adjectives in order to support automatic systems in inferencing tasks. A second contribution is a more sophisticated theory 
of the role of lexical information to inferencing. Third, an investigation of the differences between the inferencing judgments of naive speakers and those from a  
Gold standard corpora allows us to identify the pragmatic factors contributing to the interpretation of lexical items in richer linguistic contexts. 
%One PI and one consultant are women, and female and minority students will be recruited for providing Gold standard judgments. 


\vspace{-1.2em}


\paragraph{Key Words:} Inferences, lexical-semantic patterns, adjective classes, corpus data, crowdsourcing. 


\end{document}




